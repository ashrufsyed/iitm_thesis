%%% This template is meant for writing synopsis with IITM guidelines
\documentclass[PhD,synopsis,BlueTape]{iitmdiss}
%% Color choices for left band: BlueTape, YellowTape, NoColor

%% Font families
\usepackage{times}
%\usepackage[scaled]{helvet} % For using helvetica with pdftex
%\renewcommand\familydefault{\sfdefault}
%\usepackage{t1enc} % t1enc is obselete, replaced by fontenc
\usepackage[T1]{fontenc}
\usepackage[utf8]{inputenc}
\usepackage{amsmath,amssymb,amsthm,textcomp} % easier math formulae, align, subequations \ldots
%\usepackage{newtxtext,newtxmath} % can be used for getting TNR font but might mess up your other co-existing fonts/symbols
\usepackage{afterpage}
\usepackage{emptypage} % To remove page numbers on blank pages
\usepackage{authoraftertitle} %to access title and author even after maketitle using \MyTitle and \MyAuthor


\usepackage{enumerate}
\usepackage{bibentry}
\usepackage{graphicx,color,multicol}
\usepackage{epstopdf}
\epstopdfsetup{
        suffix=,
}

%% Captions, SI units and IUPAC naming
\usepackage{caption,subcaption,rotating,pdflscape}
\usepackage{siunitx,mhchem} %for SI units and IUPAC naming
%\usepackage{doi}
\captionsetup[subfigure]{aboveskip=-0.1pt,belowskip=-0.1pt}
\sisetup{detect-weight=true, detect-family=true, mode=text}
\mhchemoptions{version=4}

\usepackage{hyperref} % hyperlinks for references.
%\definecolor{myCiteColor}{RGB}{68, 118, 102}
%\usepackage[colorlinks=true,citecolor=myCiteColor]{hyperref}
%\usepackage[hypertex]{hyperref} % hypertex is no longer required option
%\usepackage[hidelinks]{hyperref} %hidelinks to remove boxes around hyperlinks


%% Optional packages for nomencl, lineno and TOC links for every page
%\usepackage{nomencl,lineno}
\usepackage{listings}


\usepackage{titletoc}
%% Optional packages for nomencl, lineno and TOC links for every page
%\usepackage{nomencl,lineno}
%\usepackage{blindtext,eso-pic,ifthen,xcolor} % For TOC link in every page%%% The following lines add Chapter or Appendix in front of the number
\titlecontents{chapter}%
[0pt]% <left>
{\vspace{4pt}}% <above-code>
{\bfseries\chapappname\ \thecontentslabel:\quad}% <numbered-entry-format>
{\bfseries}% <numberless-entry-format>
{\bfseries\mdseries\titlerule*[0.75em]{.}\contentspage}% <filler-page-format>
%{\hfill\contentspage}% <filler-page-format> %empty space option here
[\addvspace{2pt}]% <vertical-space-after-chaptername>
%%% Initially, for the main part of the document, set the label to "Chapter"
\let\chapappname\chaptername

%\usepackage{blindtext,eso-pic,ifthen,xcolor} % For TOC link in every page
%%%Alt-code if the above code fails - required packages: appendix, titletoc
% \usepackage{titletoc,appendix}
% \makeatletter
% \titlecontents{chapter}% <section-type>
% [0pt]% <left>
% {\bfseries}% <above-code>
% {\@chapapp\ \thecontentslabel:\quad}% <numbered-entry-format>
% {}% <numberless-entry-format>
% {\mdseries\titlerule*[0.75em]{.}\contentspage}% <filler-page-format> %{\hfill\contentspage}
% [\addvspace{10pt}]%
%
% \g@addto@macro\appendices{%
%       \addtocontents{toc}{\protect\renewcommand{\protect\@chapapp}{\appendixname}}%
% }
% \makeatother



%==============================================================================
\definecolor{listingcolor}{gray}{0.90}
\newcommand{\bbc}{red\xspace}

\lstset{
      language=c++, emph={fvm,div,Sp,laplacian,fvScalarMatrix,fvc,SuSp,PIMPLE, nOuterCorrectors, turbOnFinalIterOnly, ddt, turbulence, divDevReff, divDevRhoReff, solve, relax, solver, grad, magSqr, scalarField, endl, Info, U, thermo, correct, pow, extracted, fabs, vector},
      emphstyle=\textbf,
      captionpos=b,
      basicstyle=\ttfamily\footnotesize,
      showspaces=false,
      showstringspaces=true,
      extendedchars=true
}

\lstdefinestyle{std}{
      tabsize=4,
      numbers=left,
      stepnumber=1,
      numbersep=5pt,
      numberstyle=\tiny,
      breaklines=true,
      breakautoindent=true,
      postbreak=\space,
      backgroundcolor=\color{listingcolor},
      frame=lines
}

\DeclareCaptionFont{white}{\color{white}}
\DeclareCaptionFormat{listing}{\colorbox{gray}{\parbox{\textwidth}{#1#2#3}}}

%==============================================================================

\newcommand{\cref}[1]{(\ref{#1})}
%\newcommand{\rmm}{\mathrm}
%\newcommand{\mr}{\mathrm}
\newcommand{\U}{\textbf{U}}
\newcommand{\bTau}{\boldsymbol \tau}
\newcommand{\bTauInco}{\boldsymbol \tau_\mathrm{inco}}
\newcommand{\X}{\raisebox{2pt}{$\chi$}}
\newcommand{\ct}{\cellcolor{gray!10}}
\newcommand{\ctt}{\cellcolor{gray!5}}
\newcommand{\SIMPLE}{\texttt{SIMPLE}\xspace}
\newcommand{\SIMPLEC}{\texttt{SIMPLEC}\xspace}
\newcommand{\SIMPLER}{\texttt{SIMPLER}\xspace}
\newcommand{\SIMPLEM}{\texttt{SIMPLEM}\xspace}
\newcommand{\PISO}{\texttt{PISO}\xspace}
\newcommand{\PIMPLE}{\texttt{PIMPLE}\xspace}
\renewcommand{\textregistered}{\textsuperscript{\circledR}}
\newcommand\numberthis{\addtocounter{equation}{1}\tag{\theequation}}
\newcommand{\tr}{^T}
%\newcommand{\D}{\mathrm{D}}
%\newcommand{\Co}{\mathrm{Co}}
\newcommand{\vA}{\textbf{a}}
\newcommand{\vB}{\textbf{b}}
\newcommand{\vU}{\textbf{U}}
\newcommand{\tT}{\textbf{T}}
\newcommand{\OF}{OpenFOAM\textregistered\xspace}
\newcommand{\OFV}{7.x\xspace}
\newcommand{\DEV}{(23fb1cc)\xspace}
\newcommand{\red}{\color{red}}

% Operators
\newcommand{\Ddt}[1]{\frac{D #1}{dt}}
\newcommand{\E}[1]{\hbox{E}\lbrack#1 \rbrack}
\newcommand{\Var}[1]{\hbox{Var}\lbrack#1 \rbrack}
\newcommand{\Std}[1]{\hbox{Std}\lbrack#1 \rbrack}


\renewcommand{\labelenumi}{(\alph{enumi})} % Use letters for enumerate
% \DeclareMathOperator{\Sample}{Sample}
\let\vaccent=\v % rename builtin command \v{} to \vaccent{}
\renewcommand{\v}[1]{\ensuremath{\mathbf{#1}}} % for vectors
\newcommand{\gv}[1]{\ensuremath{\mbox{\boldmath$ #1 $}}}
% for vectors of Greek letters
\newcommand{\uv}[1]{\ensuremath{\mathbf{\hat{#1}}}} % for unit vector
\newcommand{\abs}[1]{\left| #1 \right|} % for absolute value
\newcommand{\avg}[1]{\left< #1 \right>} % for average
\let\underdot=\d % rename builtin command \d{} to \underdot{}
\renewcommand{\d}[2]{\frac{d #1}{d #2}} % for derivatives
\newcommand{\dd}[2]{\frac{d^2 #1}{d #2^2}} % for double derivatives
\newcommand{\pd}[2]{\frac{\partial #1}{\partial #2}}
% for partial derivatives
\newcommand{\pdd}[2]{\frac{\partial^2 #1}{\partial #2^2}}
% for double partial derivatives
\newcommand{\pdc}[3]{\left( \frac{\partial #1}{\partial #2}
        \right)_{#3}} % for thermodynamic partial derivatives
\newcommand{\ket}[1]{\left| #1 \right>} % for Dirac bras
\newcommand{\bra}[1]{\left< #1 \right|} % for Dirac kets
\newcommand{\braket}[2]{\left< #1 \vphantom{#2} \right|
        \left. #2 \vphantom{#1} \right>} % for Dirac brackets
\newcommand{\matrixel}[3]{\left< #1 \vphantom{#2#3} \right|
        #2 \left| #3 \vphantom{#1#2} \right>} % for Dirac matrix elements
\newcommand{\grad}[1]{\gv{\nabla} #1} % for gradient
\let\divsymb=\div % rename builtin command \div to \divsymb
\renewcommand{\div}[1]{\gv{\nabla} \cdot #1} % for divergence
\newcommand{\curl}[1]{\gv{\nabla} \times #1} % for curl
\let\baraccent=\= % rename builtin command \= to \baraccent
\renewcommand{\=}[1]{\stackrel{#1}{=}} % for putting numbers above =
\newtheorem{prop}{Proposition}
\newtheorem{thm}{Theorem}[section]
\newtheorem{lem}[thm]{Lemma}
\theoremstyle{definition}
\newtheorem{dfn}{Definition}
\theoremstyle{remark}
\newtheorem*{rmk}{Remark}
%==============================================================================
                                             


% \usepackage{times}
% %\usepackage{t1enc} % t1enc is obselete, replaced by fontenc
% \usepackage[T1]{fontenc}
% %\usepackage[dvips]{graphicx}
% \usepackage{hyperref} % hyperlinks for references.
% %\usepackage[hypertex]{hyperref} % hypertex is no longer required option
% %\usepackage[hidelinks]{hyperref} %hidelinks to remove boxes around hyperlinks
% \usepackage{amsmath,amssymb,textcomp} % easier math formulae, align, subequations \ldots
% \usepackage{graphicx}
% \usepackage{epstopdf}

\title{A New and Improved \LaTeX\ Class for Dissertations Submitted to IIT Madras}

\author{NAME}

\date{July 2021}
\department{AEROSPACE ENGINEERING}

\begin{document}
%\nocite{*}
\newgeometry{top=2cm, right=1.27cm, bottom=1.5cm}
\maketitle
\restoregeometry

% The main text will follow from this point so set the page numbering
% to arabic from here on.
\pagenumbering{arabic}
\setcounter{page}{0}
%\newpage

%%%  Single spacing for the entire synopsis report
\singlespacing

The style of preparation (fonts, Chapter No., margins, section and subsection no.,
equation no., figures, tables, etc.) must be as in case of the thesis. Ph.D. synopsis should be 4-6 pages long and M.S. synopsis should be 2-3 pages
long, excluding the cover page. Single-line spacing and 12-pt font size should be used in the synopsis. References should be given in 10-pt font size. Here are some examples for citing references: \cite{lamport:86} and \cite{Ahren2005} along with \cite{Roenby2016}. Check the official guidelines for more details.

\section{Abstract}
Brief outline of the reported research work

\section{Objectives}
State clear objectives of the research work

\section{Existing Gaps Which Were Bridged}
Trace to the point, the research gaps that were bridged in the work

\section{Most Important Contributions}
Briefly emphasize the importance of the research work. Do not necessarily try to elaborate in a detailed manner.

\section{Conclusions}
Highlight major (and not all) conclusions.


%%%%%%%%%%%%%%%%%%%%%%%%%%%%%%%%%%%%%%%%%%%%%%%%%%%%%%%%%%%%
% Bibliography.

%\begin{singlespace}
%\begin{thebibliography}{10}
%\addcontentsline{toc}{section}{Bibliography}
%
%\bibitem{paper1}
%Author 1 and Author 2
%\newblock {\em Paper title}.
%\newblock Journal\ \ {\bf Volume}, Page\ \ (Year).
%\end{thebibliography}

%\end{singlespace}

%%%  Older format  %%%
%\section{Proposed Contents of the Thesis}
%The outline of the thesis is as follows:
%\begin{enumerate}
%\item Chapter 1 title
%\item Chapter 2 title
%\item ...
%\end{enumerate}


%\vskip 4cm
%%%%%%
\section{List of Publications}
List of DOIs of any published or presented work

%%%  Older format
%\subsection{Papers in Refereed Journals}
%\begin{enumerate}
%\item Title \\
%	{\bf Author 1}, Author 2...\\
%	{\it Journal title.}, {\bf Volume}, Page (Year)
%\end{enumerate}
%
%\subsection{Presentations in Conferences}
%\begin{enumerate}
%\item  Presented titled  {\em Title} at the {\bf Conference on...}  place, date..
%\end{enumerate}

\small %To make font size 10
	\bibliography{refs}


%%%%%%%%%%%%     References     %%%%%%%%%%%%%%
%% References in the synopsis can be added by either of the two methods: 1) the simpler method which directly uses a .bib file which is an entire library for all the references 2) the not-so-easy process where generating the bib items format (.bbl) file using the .bib library, and then adding the generated .bbl files and recompiling to get the final document

%%% Detailed procedure for the second method (useful for generating your list of publications using bib)
%%% After adding all the references to the document, compile the tex file with \bibliography{refs} like you did for thesis which uses the refs.bib, the entire library file
%%% This generates a synonpsis.bbl file which has the references only from the document, and in bibitem format
%%% Copy the entire text which starts with \begin{thebibliography}, and ends with \end{thebibliography}, and then recompile by commenting out the original \bibliography{refs} line. You might have to recompile few times to get the correct author-year format

%% Below is an example of bbl file that has been generated for the present document
%\begin{thebibliography}{3}
%	\expandafter\ifx\csname natexlab\endcsname\relax\def\natexlab#1{#1}\fi
%	\expandafter\ifx\csname url\endcsname\relax
%	\def\url#1{{\tt #1}}\fi
%	\expandafter\ifx\csname urlprefix\endcsname\relax\def\urlprefix{URL }\fi
%	
%	\bibitem[{Ahren {\em et~al.\/}(2005)Ahren, Gevci, and Law}]{Ahren2005}
%	{\bf Ahren, J.}, {\bf B.~Gevci}, and {\bf C.~Law} (2005).
%	\newblock {ParaView: An End-User Tool for Large-Data Visualization}.
%	\newblock {\em In\/} {\em Visualization Handbook\/}, 717--731. Elsevier.
%	\newblock ISBN 978-0123875822.
%	\newblock \href{https://doi.org/10.1016/B978-012387582-2/50038-1}{\tt
%		doi:10.1016/B978-012387582-2/50038-1}.
%	
%	\bibitem[{Lamport(1986)}]{lamport:86}
%	{\bf Lamport, L.} (1986).
%	\newblock {\em \LaTeX: A document preparation system\/}.
%	\newblock Addision-Wesley.
%	
%	\bibitem[{Roenby {\em et~al.\/}(2016)Roenby, Bredmose, and Jasak}]{Roenby2016}
%	{\bf Roenby, J.}, {\bf H.~Bredmose}, and {\bf H.~Jasak} (2016).
%	\newblock {A computational method for sharp interface advection}.
%	\newblock {\em Royal Society Open Science\/}, {\bf 3}(11), 160405.
%	\newblock ISSN 2054-5703, \href{https://doi.org/10.1098/rsos.160405}{\tt
%		doi:10.1098/rsos.160405}, \href{https://www.arxiv.org/abs/1601.05392v2}{\tt
%		arXiv:1601.05392v2}.
%	
%\end{thebibliography}
\end{document}

